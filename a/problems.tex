\documentclass[12pt]{ctexart}
\title{\textbf{NOIP模拟赛}}
\usepackage{amsmath}
\author{GGN\&HJQ}
\date{2020年8月9日}
\begin{document}
\maketitle
\begin{center}
	\begin{tabular}{|c|c|c|c|}
		\hline 题目名称&crazy typer&treat guest&qwq similarity\\
		\hline 源文件名&typer.cpp&guest.cpp&qwq.cpp\\
		\hline 输入文件名&typer.in&guest.in&qwq.in\\
		\hline 输出文件名&typer.out&guest.out&qwq.out\\
		\hline 题目类型&传统&传统&传统\\
		\hline 时间限制&1s&1s&2s\\
		\hline 空间限制&512MB&512MB&512MB\\
		\hline 测试点数目&10&10&10\\
		\hline 每个测试点分值&10&10&10\\
		\hline 编译选项&-lm&-lm&-lm\\
		\hline
	\end{tabular}
\end{center}
\textbf{注意事项}\\
1.评测方式:全文比较,忽略行末空格和文末回车\\
2.请不要直接从题面中复制样例
\newpage
\section{crazy typer}
\subsection{题目背景}
GGN是一个不喜欢出题的女孩子。为什么GGN不喜欢出题呢?是因为她不喜欢打字。为什么GGN不喜欢打字呢?是因为她打字太慢。直到有一天,F老师让GGN出一套难度适中的模拟赛,要求题面总字数尽可能多,机智的GGN决心通过“复制-粘贴”这一传统的方法凑够题面字数。
\subsection{题目描述}
起初,GGN的文件中只有一个字符,剪贴板中没有字符。每个时刻,GGN只能进行如下两种操作:\\
1.“全选-复制”,即将文件中的内容全部拷贝进入剪贴板。\\
2.“粘贴”,即在文件尾部追加剪贴板中的内容,但并不改变剪贴板中的内容。\\

形式化的表述如下:
假设GGN某次操作之前,文件中字符总数为$X$,剪贴板中字符总数为$Y$。执行这次操作后,文件中字符总数记为$X'$,剪贴板中字符总数记为$Y'$。\\
若这次操作为“操作1”,则有$X'=X,Y'=Y$。若这次操作为“操作2”,则有$X'=X+Y,Y'=Y$。\\

由于时间有限,所以她只能做$N$次操作。已知GGN的总操作次数为$N$,起初文章中有一个字符,剪贴板中没有字符。她希望你能帮她求出,进行$N$次操作后,文件中字符总数可能达到的最大值$T$。由于答案可能很大,而GGN又不喜欢写高精度,所以你只需要求出$T$除以$998244353$的余数。
\subsection{输入格式}
共一行,含一个整数$N$,表示GGN操作的总次数。
\subsection{输出格式}
共一行,含一个整数,即$T$除以998244353的余数。
\subsection{输入输出样例}
\begin{center}
	\begin{tabular}{|p{6cm}|p{6cm}|}
		\hline typer.in&typer.out\\
		\hline7&12\\
		\hline
	\end{tabular}
\end{center}
\subsection{样例解释}
在此给出一种可能的操作方式,操作序列以及每次操作后文件和缓冲区的字符数如下:
\begin{center}
	\begin{tabular}{|p{4cm}|p{4cm}|p{4cm}|}
		\hline 操作&文件&缓冲区\\
		\hline 操作1&1&1\\
		\hline 操作2&2&1\\
		\hline 操作2&3&1\\
		\hline 操作2&4&1\\
		\hline 操作1&4&4\\
		\hline 操作2&8&4\\
		\hline 操作2&12&4\\
		\hline
	\end{tabular}
\end{center}
\subsection{数据范围}
\noindent 对于30\%的数据,保证$N\leq25$。\\
对于50\%的数据,保证$N\leq500$。\\
对于80\%的数据,保证$N\leq10^5$。\\
对于100\%的数据,保证$N\leq10^{12}$。

\newpage
\section{treat guest}
\subsection{题目背景}
“在那座阴雨的小城里我从未忘记你。”——《成都》

GGN是一个热情好客的女孩子,她经常请她的男神YZB到家里玩耍。GGN家所在城市潮湿多雨,所以道路上经常会有积水。由于YZB非常的英勇,所以他是不怕路面积水的。但是GGN非常胆小,所以她不能通过有积水的路面。因此,当YZB要通过一条有积水的马路时,GGN就会对他说:“送君千里,终有一别。”然后转身离去,消失在暮色中,此后YZB就迎来了独自一人的旅程。由于GGN是个很贴心的女孩子,所以她希望YZB独自走过的路程尽可能短。此题由此展开。
\subsection{题目描述}
GGN和YZB家所在的城市可以用一张无向图表示,图中有$N$个点和$M$条边。其中,1号结点表示GGN的家,$N$号结点表示YZB的家。每条边表示一条可供行走的路,每条边都有两个属性$hight$和$len$,分别表示这条路的海拔和长度(这两条属性不随时间推移而改变)。一条路径的总路程即这条路径上所有边的$len$属性值之和。每天,GGN和YZB都会从GGN的家(1号结点)出发,在不经过有积水的边的条件下到达某个结点,此后YZB会从这个结点独自走到自己的家($N$号结点),YZB独自行走时可以通过任何一条边。每天,给出城市中整体的水位$S$。如果一条边的$height$属性值大于等于$S$,那么在这一天中,这条边是没有积水的;如果一条边的$height$属性小于$S$,那么在这一天中,这条边是有积水的,GGN不能通过这条边。给出图的所有信息、总天数、以及每天的积水深度,计算每天YZB独自走过总路程的最小值。特殊地,如果存在一条从1号结点到$N$号结点的路径,满足这条路径上的所有的边都没有积水,那么YZB独自走过总路程的最小值为0。
\subsection{输入格式}
\noindent 第一行,包含两个整数$N,M$,表示图中结点数量以及边的数量。\\
接下来的$M$行,每行四个整数$from_i,to_i,height_i,len_i$,分别表示这条无向边所连接的两个点以及这条边的海拔和长度。\\
接下来1行,包含一个正整数$D$,表示总天数。\\
接下来1行,$D$个整数,其中第$i$个数表示第$i$天的积水深度$S_i$。
\subsection{输出格式}
共一行,包含$D$个整数,其中第$i$个整数表示第$i$天YZB独自走过的总路程的最小值。
\subsection{输入输出样例}
\begin{center}
	\begin{tabular}{|p{6cm}|p{6cm}|}
		\hline guest.in&guest.out\\
		\hline5 6&0 6 10 12\\
				1 2 3 1&\\
				2 3 3 2&\\
				3 4 2 4&\\
				4 5 1 6&\\
				1 3 3 3&\\
				4 2 2 5&\\
				4&\\
				1 2 3 4&\\
		\hline
	\end{tabular}
\end{center}
\subsection{样例解释}
第一天,所有的边都没有积水,YZB独自走过的最短路程长度是0。

第二天,GGN先陪YZB走$1\to3\to4$,然后YZB独自走$4\to5$,独自走过的总路程为6(方案不唯一)。

第三天,GGN先陪YZB走$1\to3$,然后YZB独自走$3\to4\to5$,独自走过的总路程为10。

第四天,所有的道路都有积水,YZB独自走$1\to3\to4\to5$,独自走过的总路程为12。
\subsection{数据范围}
\paragraph{注:“无”表示无除 100\% 数据范围约束外的其他约束。}
\begin{center}
	\begin{tabular}{|c|c|c|c|c|}
		\hline 测试点编号&n&m&$height_i$&$S_i$\\
		\hline 1&$\le500$&$\le1000$&$\le100$&无\\
		\hline 2&$\le500$&$\le1000$&$\le200$&无\\
		\hline 3&$\le500$&$\le1000$&$\le300$&无\\
		\hline 4&$\le500$&$\le1000$&$\le400$&无\\
		\hline 5&$\le500$&$\le1000$&$\le500$&无\\
		\hline 6&$=6250$&$=9793$&无&无\\
		\hline 7&无&无&1&无\\
		\hline 8&无&无&$\le10000$&$\le10000$\\
		\hline 9,10&无&无&无&无\\
		\hline
	\end{tabular}
\end{center}
对于100\%的数据,$2\leq N\leq10^5,1\leq M\leq4\times10^5,1\leq D\leq10^5,0<height_i<2^{31}-1,0<len_i<10^7$。

\newpage
\section{qwq similarity}
\subsection{题目背景}
HJQwQ不喜欢出题,不热情好客,也不是女孩子,但他喜欢打MC。一天,F老师让他出一套难度适中的模拟赛。但HJQwQ并不想出题,于是就随便抄了几道原题,然后就去打MC了。F老师知道之后非常不爽,于是她想出了题目查重的办法:将每道题目看成一个序列。她认为一个题目(一个短序列)和题库(一个长序列)的最长公共\textbf{子序列}越长,这道题目的抄袭程度就越高。她想知道对于每道题,HJQwQ到底抄袭了多少?
\subsection{题目描述}
我们将问题简化为:给你一个长为$N$的序列$S$,有$M$次询问,第i次询问会给出一个序列$T_i$,你需要回答$S$与$T_i$的最长公共\textbf{子序列(注意不是子串)}的长度。
\subsection{输入格式}
\noindent 第1行2个用空格分隔的正整数$N,M$\\
第2行$N$个用空格分隔的正整数$S_i$\\
第$3\sim(M+2)$行,每行第1个正整数表示$T_i$的长度$L_i$,其后有$L_i$个用空格分隔的正整数$T_{ij}$
\subsection{输出格式}
共$M$行,第$i$行输出一个非负整数$ans_i$表示$S$与$T_i$的最长公共\textbf{子序列}的长度,若没有公共子序列则输出0
\subsection{输入输出样例}
\subsubsection{样例1}
\begin{center}
	\begin{tabular}{|p{6cm}|p{6cm}|}
		\hline qwq.in&qwq.out\\
		\hline  10 5&1\\
				5 1 4 3 2 6 5 5 1 7&0\\
				1 5&4\\
				1 10&3\\
				4 5 1 4 7&5\\
				5 4 1 2 3 5&\\
				5 2 6 5 1 7&\\
		\hline
	\end{tabular}
\end{center}
见下发文件的qwq1.in和qwq1.out
\subsubsection{样例解释}
\noindent 第一次询问的答案为1,一种方案为[5]\\
第二次询问的数10未在$S$中出现过,故答案为0\\
第三次询问的答案为4,一种方案为[5,1,4,7]\\
第四次询问的答案为3,一种方案为[4,2,5]\\
第五次询问的答案为5,一种方案为[2,6,5,1,7]
\subsubsection{样例2}
见下发文件的qwq2.in和qwq2.out,此样例数据范围与测试点1相同
\subsubsection{样例3}
见下发文件的qwq3.in和qwq3.out,此样例数据范围与测试点3相同
\subsubsection{样例4}
见下发文件的qwq4.in和qwq4.out,此样例数据范围与测试点9相同
\subsection{数据范围}
\begin{center}
	\begin{tabular}{|c|c|c|c|c|}
		\hline 测试点编号&$N$&$M$&$L_i$&$S_i,T_{ij}$\\
		\hline 1,2&$\leq10$&$\leq100$&$\leq5$&$\leq10^6$\\
		\hline 3,4,5&$\leq1000$&$\leq1000$&$\leq20$&$\leq10^6$\\
		\hline 6&$\leq10^6$&$\leq10000$&$=1$&$\leq10^6$\\
		\hline 7&$\leq10^6$&$\leq10000$&$\leq 5$&$\leq10^6$\\
		\hline 8&$\leq10^6$&$\leq10000$&$\leq 20$&$\leq2$\\
		\hline 9,10&$\leq10^6$&$\leq10000$&$\leq 20$&$\leq10^6$\\
		\hline
	\end{tabular}
\end{center}
对于100\%的数据,有$1\le N,S_i,T_{ij}\le10^6,1\le M\le10000,1\le L_i\le20$
\end{document}