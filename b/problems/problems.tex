\documentclass[12pt]{ctexart}
\usepackage{amsmath}
\usepackage{wasysym}
\title{\textbf{NOIP模拟赛2}}
\author{GGN\&HJQ}
\date{2020年8月16日}
\begin{document}
\maketitle
\begin{center}
	\begin{tabular}{|c|c|c|c|}
		\hline 题目名称&多米诺骨牌&抽卡&归程\\
		\hline 英文名称&domino&card&return\\
		\hline 源文件名&domino.cpp&card.cpp&return.cpp\\
		\hline 输入文件名&domino.in&card.in&return.in\\
		\hline 输出文件名&domino.out&card.out&return.out\\
		\hline 题目类型&传统&传统&传统\\
		\hline 时间限制&1s&1s&1s\\
		\hline 空间限制&512MB&512MB&512MB\\
		\hline 测试点数目&10&10&10\\
		\hline 每个测试点分值&10&10&10\\
		\hline
	\end{tabular}
\end{center}
\textbf{注意事项}\\
1.评测环境:ubuntu18.04 lts 64位,CPU为Intel Core i7-8550U\\
2.评测软件:lemon\\
3.编译工具:g++10.1.0(如果担心Compile Error可以先发给cppascalinux进行编译测试)\\
3.编译命令:g++ -o \%s \%s.cpp -lm (\%s为题目英文名)\\
4.比较方式:全文比较,忽略行末空格和文末回车\\
5.请不要直接从题面中复制样例
\newpage
\section{多米诺骨牌}
\subsection{题目背景}
HJQwQ在玩多米诺骨牌,但他已经玩腻了摆放然后推倒的玩法.于是他找来了国际象棋棋盘,开始往上面摆骨牌.这时他突发奇想:棋盘上最多能摆多少个不相邻的骨牌?但是他还要去打MC,于是把这个问题留给了你
\subsection{题目描述}
有一个$n$行$m$列($n*m$)的棋盘,在它的上面放若干张1*2或2*1的骨牌,要求骨牌必须完全位于棋盘内,每个格子至多只被一张骨牌覆盖,且这些骨牌两两不相邻(我们定义两张骨牌是"相邻"的,当且仅当这两个骨牌至少有一条公共边或一个公共顶点,如图1-1和1-2中的两个骨牌是相邻的,而1-3中的两个骨牌不相邻),求最多可以摆放骨牌的数目
\subsection{输入格式}
第一行一个正整数$T$,表示数据组数
接下来$T$行,每行两个正整数$n,m$
\subsection{输出格式}
共$T$行,每行一个整数,表示对于每组数据的答案
\subsection{输入输出样例}
\begin{center}
	\begin{tabular}{|p{6cm}|p{6cm}|}
		\hline domino.in&domino.out\\
		\hline	5&1\\
				1 2&2\\
				3 2&5\\
				5 5&5\\
				4 6&9\\
				5 8&\\
		\hline
	\end{tabular}
\end{center}
\subsection{样例解释}
对于5组数据,各给出一种可行的方案(图2-1$\to$2-5),不难证明都是最大方案
\subsection{数据范围}
\begin{center}
	\begin{tabular}{|c|c|c|}
		\hline 测试点编号&$n$&$m$\\
		\hline 1&$=1$&$\le10$\\
		\hline 2,3&$\le2$&$\le10^5$\\
		\hline 4,5&$\le5$&$\le10^5$\\
		\hline 6,7&$\le10$&$\le1000$\\
		\hline 8,9,10&$\le10^5$&$\le10^5$\\
		\hline
	\end{tabular}
\end{center}
对于100\%的数据,有$1\le T\le10,1\le n,m\le10^5$
\newpage
\section{抽卡}
\subsection{题目背景}
HJQwQ在和GGN玩一个抽卡游戏.他们的面前摆放着$n$张背面朝上的卡,每张卡的正面印着一个价值$v_i$.HJQwQ可以从中抽走$k$张,并获得这些卡上所有的价值.擅长出老千的HJQwQ早就看穿了每张卡的价值,然而机智的GGN更早就料到HJQwQ会出老千,于是他又加了一条规则:取出的k张卡的位置必须两两不相邻.这下HJQwQ方了:他该抽走哪些卡,才能使的自己获得的总价值最大?
\subsection{题目描述}
有一个长为$n$的数组$v_i$,请从中选出$k$个不相邻的位置,使得这些位置的和最大,并输出这个和,你需要对$k\in\left[1,\lfloor\frac{n+1}{2}\rfloor\right]$都输出一个答案
\subsection{输入格式}
第一行一个正整数$n$

第二行$n$个用空格分隔的正整数$v_1\sim v_n$
\subsection{输出格式}
共$\lfloor\frac{n+1}{2}\rfloor$行,每行一个整数,第$i$行的整数表示$k=i$时的答案
\subsection{样例}
\begin{center}
	\begin{tabular}{|p{6cm}|p{6cm}|}
		\hline card.in&card.out\\
		\hline	5&5\\
				1 5 3 3 4&9\\
				&8\\
		\hline
	\end{tabular}
\end{center}
\subsection{样例解释}
$k=1$时,抽$v_2=5$最大

$k=2$时,抽$v_2+v_5=9$最大

$k=3$时,抽$v_1+v_3+v_5=8$最大
\subsection{数据范围}
\begin{center}
	\begin{tabular}{|c|c|c|}
		\hline 测试点编号&$n$&$v_i$\\
		\hline 1,2&$\le10$&$\le10^9$\\
		\hline 3,4,5&$\le2000$&$\le10^9$\\
		\hline 6,7&$\le2\times10^5$&$\le2$\\
		\hline 8,9,10&$\le2\times10^5$&$\le10^9$\\
		\hline
	\end{tabular}
\end{center}
对于100\%的数据,有$1\le n\le2\times10^5,1\le v_i\le10^9$
\newpage
\section{归程}
\subsection{题目背景}
HJQwQ在地下挖了一背包的铁和钻石,打算坐矿车回家,原本的铁路是一条直线.但命运之神GGN和他开了一个小玩笑:他使用魔法,把HJQwQ的铁路变成了一张有向无环图! HJQwQ在每一个岔道口处,会随机地沿着一条出边前进,只有当一个岔道口没有出边,他才可以停下来.

HJQwQ自闭了,但他也不是无计可施:他可以使用魔法,但由于他的法力太弱,只能从图中删除至多一条边(也可以不删).他想知道,怎样才能使旅程尽快结束(期望意义下)?
\subsection{题目描述}
形式化的表述如下:有一张$n$个点,$m$条边的有向无环图$G(V,E)$,每条边$(u,v)$有一个边权$w_{(u,v)}$和一个长度$l_{(u,v)}$\textbf{(可能为负)},若HJQwQ在某个点$u$,则他选择边$(u,v)$的概率为$\frac{w_{(u,v)}}{\sum\limits_{(u,i)\in E}w_{(u,i)}}$,即当前出边的权值除以所有出边权值之和.

HJQwQ从1号点出发,按照这个规则不断前进,直到一个没有出边的点才停止,他经过的总路程为他经过的边的长度$l(u,v)$之和.在出发之前,他可以选择一条边$(u,v)$,并从图中删去这条边(也可以不选).请求出他经过的总路程长度的期望值最小是多少.
\subsection{输入格式}
第1行两个正整数$n,m$,分别表示图中的点数和边数

第$2\sim(m+1)$行,每行4个整数$u,v,w,l$,表示有一条从$u$到$v$的有向边,权值为$w$,长度为$l$
\subsection{输出格式}
一行共一个有理数,表示总路程的期望值的最小值,结果四舍五入保留三位小数
\subsection{输入输出样例}
\subsubsection{样例1}
\begin{center}
	\begin{tabular}{|p{6cm}|p{6cm}|}
		\hline return.in&return.out\\
		\hline	5 6&5.600\\
				1 2 1 3&\\
				1 3 2 4&\\
				2 3 3 5&\\
				2 4 4 1&\\
				3 4 2 3&\\
				3 5 1 2&\\
		\hline
	\end{tabular}
\end{center}
见下发文件的return1.in和return1.out
\subsubsection{样例解释}
\begin{center}
	\begin{tabular}{|c|c|}
		\hline 删除边的编号(0表示不删边)&总路程期望(保留三位小数)\\
		\hline 0&6.514\\
		\hline 1&6.400\\
		\hline 2&6.743\\
		\hline 3&5.600\\
		\hline 4&7.733\\
		\hline 5&6.190\\
		\hline 6&7.000\\
		\hline
	\end{tabular}
\end{center}
\subsubsection{样例2}
见下发文件的return2.in和return2.out,此样例数据范围与测试点1相同
\subsubsection{样例3}
见下发文件的return3.in和return3.out,此样例数据范围与测试点3相同
\subsubsection{样例4}
见下发文件的return4.in和return4.out,此样例数据范围与测试点8相同
\subsection{数据范围}
\begin{center}
	\begin{tabular}{|c|c|c|}
		\hline 测试点编号&$n$&$m$\\
		\hline 1,2&$\le10$&$\le45$\\
		\hline 3,4,5&$\le1000$&$\le2000$\\
		\hline 6,7&$\le2\times10^5$&$=n-1$\\
		\hline 8,9,10&$\le2\times10^5$&$\le4\times10^5$\\
		\hline
	\end{tabular}
\end{center}
对于100\%的数据,有$1\le n\le2\times10^5,1\le m\le4\times10^5,1\le w\le10,-10\le l\le10$,保证图为有向无环图,且图内无重边,自环
\subsection{提示}
本题输入数据量较大,请使用快速的读入方式
\end{document}